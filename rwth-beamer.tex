% This example is meant to be compiled with lualatex or xelatex
% The theme itself also supports pdflatex
\PassOptionsToPackage{unicode}{hyperref}
\documentclass[aspectratio=1610, 17pt]{beamer}

% Load packages you need here
\usepackage{polyglossia}
\usepackage{beamerappendixnote}
\usepackage{xfrac}
\usepackage{booktabs}
\usepackage{multirow}
\usepackage{tikz}
\setmainlanguage{german} % Change this to your language
% \setbeamerfont{button}{size=\scriptsize}
\setbeamertemplate{button}{\tikz\node[inner xsep = 5pt,
draw = structure!90,
fill=structure!40, 
rounded corners = 5pt]{\scriptsize\insertbuttontext};}
\setbeamercolor{button}{bg=rwthdarkblue,fg=black}

\let\oldFootnote\footnote
\newcommand\nextToken\relax

\renewcommand\footnote[1]{%
    \oldFootnote{#1}\futurelet\nextToken\isFootnote}

\newcommand\isFootnote{%
    \ifx\footnote\nextToken\textsuperscript{,}\fi}

\usepackage{csquotes}


\usepackage{amsmath}
\usepackage{amssymb}
\usepackage{mathtools}

\usepackage{hyperref}
\usepackage{bookmark}
\usepackage[
  locale=DE,
  separate-uncertainty=true,
  per-mode=symbol-or-fraction,
]{siunitx}
\usepackage{subcaption}

% Captions schöner machen.
\usepackage[
  font=small,          % Schrift etwas kleiner als Dokument
  width=0.9\textwidth, % maximale Breite einer Caption schmaler
  skip=3pt,            % Abstand zwischen Caption und Text, 10pt ist default
]{caption}

\usepackage[
  version=4,
  math-greek=default, % ┐ mit unicode-math zusammenarbeiten
  text-greek=default, % ┘
]{mhchem} % chemische Formeln, zur Notation von Isotopen: \ce{^{227}_{90}Th+}

% load the theme after all packages

\usetheme[
  % showtotalframes, % show total number of frames in the footline
  % dark, % optional dark theme, uncomment to use
]{rwth}

% Put settings here, like
\unimathsetup{
  math-style=ISO,
  bold-style=ISO,
  nabla=upright,
  partial=upright,
  mathrm=sym,
}

\newcommand{\source}[1]{{\centering\caption*{\scriptsize\textcolor{tugreen}{Quelle:} \centering {#1}}}}
\newcommand{\dd}{\symup{d}}

\title{Example title} % Put your title here
\subtitle{Subtitle} % Put your subtitle here, leave empty to not use a subtitle
\author[J.~Minor]{Jonas Minor} % Put your name here
\date{24. June 2042} % Put the date here
\institute[]{Simulation Sciences} % Put your institute here
% \titlegraphic{\includegraphics[width=0.5\textwidth]{images/Cover.png}} % Put the titlegraphic here

\begin{document}

\maketitle

\begin{frame}{Motivation}
  \begin{itemize}
    \item Example
    \item itemize 
          \begin{itemize}
            \item Sub
            \item Points
          \end{itemize}
    \item End
  \end{itemize}
\end{frame}

\begin{frame}{Outlook}
  \begin{itemize}
        \item Some final Points
  \end{itemize}
  \vspace{0cm}
  \centering{\only<1>{}\only<2>{\Large{\color{rwthdarkblue}{Thank you for your attention \\ Any questions?}}}} % Only is used to hide the text when the slide is shown first
\end{frame}

\begin{frame}
  \centering
  \Huge{\color{rwthdarkblue}{\textbf{Backup}}}
\end{frame}

\printappxnotes

\begin{frame}{Example Appendix}
  \vspace{-1ex}
  \small
  \begin{equation*}
    \frac{\dd E}{\dd x} = {\left(\frac{\dd E}{\dd x}\right)}_\text{Ion} + {\left(\frac{\dd E}{\dd x}\right)}_\text{Brems}
  \end{equation*}
  \begin{equation*}
    - {\left(\frac{\dd E}{\dd x}\right)}_\text{Ion} = 2 \pi N_\text{A} r_\text{e}^2 m_\text{e} c^2 \rho \frac{Z}{A} \frac{1}{\beta^2} \left[ \!\ln \left(\frac{\tau^2 \left(\tau + 2\right)}{2 {\left(\frac{I}{m_\text{e}}\right)}^2}\right) - F \!\left(\tau\right) - \delta - 2\frac{C_\text{s}}{Z} \right]
  \end{equation*}
  \begin{center}
    \vspace{-0em}
    \footnotesize
    $\tau$: kinetic energy of $m_\text{e} c^2$, $\rho$: density, $I$: average excitation potential, $C_\text{s}$: correction, $F(\tau) = 1 - \beta^2 + \frac{\tau^2 - 8 \left(2 \tau + 1\right) \ln \left(2\right)}{8 {\left(\tau + 1\right)}^2}$
  \end{center}
  \begin{equation*}
    - {\left(\frac{\dd E}{\dd x}\right)}_\text{Brems} = \frac{4}{137} r_\text{e}^2 N_\text{A} \frac{Z^2}{A} m_\text{e} c^2 \tau \left(\ln \!\left(2 \left(\tau + 1\right)\right) - \sfrac{1}{3}\right)
  \end{equation*}
\end{frame}

\end{document}
